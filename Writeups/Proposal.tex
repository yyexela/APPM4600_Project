\documentclass{article}

% Maths
\usepackage{amsmath}

% Hyperlink
\usepackage{hyperref}
\hypersetup{
    colorlinks=true,
    linkcolor=blue,
    filecolor=magenta,      
    urlcolor=cyan,
    pdfpagemode=FullScreen,
}

% Title
\title{Project Proposal}
\author{Alexey Yermakov\\Logan Barnhart\\Tyler Jensen}

% Figures
\usepackage{graphics, float, subfig}
\usepackage[pdflatex]{graphicx}

% Itemize
\renewcommand{\labelitemi}{\textbullet}
\renewcommand{\labelitemii}{\textbullet}
\renewcommand{\labelitemiii}{\textbullet}

% Margins
\usepackage[margin=1in]{geometry}

% Sections
\setcounter{secnumdepth}{0}

% Helpful commands
\newcommand{\x}{\mathbf{x}}
\newcommand{\A}{\mathbf{A}}
\newcommand{\B}{\mathbf{b}} % \b already defined
\newcommand{\I}{\mathbf{I}}

\begin{document}

    \maketitle

    \section{Introductory Material}

    We have decided to work on Project 3: Regularization in Least Squares. The introductory material for this project includes matrices, unitary matrices, matrix operations, data fitting, and least squares. We will also elaborate on why least squares is useful in the context of noise and data.

    \section{Distribution of Work}

    For the independent part of the project we plan to extend the notion of discrete least squares regularization to regularized continuous least squares.

    In order to meet the project's requirements of having everyone work on each of:

    \begin{itemize}
        \item Researching
        \item Writing
        \item Coding
        \item Presentation
    \end{itemize}

    we will assign work in the following way:

    \subsection{Ridge Regression}

    \begin{itemize}
        \item 2.1.1: Deriving the Ridge Estimator - Alexey (Research)
        \item 2.1.2: Exploring the Ridge Estimator - Tyler (Coding)
    \end{itemize}

    \subsection{Tikhonov Regularization}

    \begin{itemize}
        \item 2.2.1: Show that $\textbf{D}$ is unitary - Tyler (Research)
        \item 2.2.2: Tikhonov regularization - Logan (Research)
        \item 2.2.3: Tikhonov modeling - Logan (Coding)
        \item 2.2.4: Other finite difference method - Alexey (Coding)
    \end{itemize}

    \subsection{Independent Extension}

    For the independent extension we'll explore LASSO and elastic net for the least squares regularization. Whereas we traditionally we can write the least squares problem as:

    \begin{equation*}
        ||y-X\beta||^{2}_{2} + \lambda_2 ||\beta||_{2}^{2}
    \end{equation*}

    The LASSO method is similar, except the regularization term has an L1 norm:

    \begin{equation*}
        ||y-X\beta||^{2}_{2} + \lambda_1 ||\beta||_{1}
    \end{equation*}

    The elastic net method combines the previos two methods by having both regularization terms in summation:

    \begin{equation*}
        ||y-X\beta||^{2}_{2} + \lambda_1 ||\beta||_{1} + \lambda_2 ||\beta||_{2}^{2}
    \end{equation*}

    We'll explore these methods by doing the following:

    \begin{itemize}
        \item 3.1 Explore the geometric interpretation of LASSO and how it affect the weights of a model and compare it to Ridge Regression and Tikhonov - Alexey/Logan/Tyler (Research)
        \item 3.2 Implement an algorithm that fits LASSO since there's no closed form formula (using something like coordinate descent) and compare it to Ridge Regression and Tikhonov - Alexey/Logan/Tyler (Coding)
        \item 3.3 Build a simple regression model using LASSO and fit to noisy data  - Alexey/Logan/Tyler (Coding)
        \item 3.4 Implement an algorithm that fits elastic net since there's no closed form formula (using something like coordinate descent) and compare it to Ridge Regression and Tikhonov - Alexey/Logan/Tyler (Coding)
        \item 3.5 Build a simple regression model using elastic net and fit to noisy data  - Alexey/Logan/Tyler (Coding)
    \end{itemize}

    Seeing as how this is a group project, the above assignments aren't strict and group members are encouraged to collaborate with one another.

    \section{Milestones}

    Given the distribution of work above, we have set the following milestones for our project:

    \begin{itemize}
        \item Finish \textbf{Ridge Regression}: November 5
        \item Finish \textbf{Tikhonov Regularization}: November 15
        \item Finish \textbf{Independent Extension} December 14
    \end{itemize}

    \section{Deadlines}

    The various deadlines we have are:

    \begin{itemize}
        \item 2-page proposal (October 29)
        \item Rough draft (November 15)
        \item Final draft (December 14)
        \item Presentation (December 14)
    \end{itemize}

\end{document}
